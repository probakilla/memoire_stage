\section{Présentation de l'entreprise} \label{sec:presentation_entreprise}

Alltech est une agence de conseil en transformation digitale créée en 2015 par David Claverie (Président) et Frédéric 
Lascombe (Directeur général). Très vite, Wilfrid Picq les rejoint et devient directeur associé. Alltech a pour vocation 
de promouvoir les outils numériques, de développer des solutions basées sur ces outils et de permettre aux entreprises 
de les exploiter. Elle est composée de consultants en informatique (qui travaillent au forfait, en TMA ou en Assistance 
Technique) ainsi que de commerciaux chargés de cibler les entreprises ayant des besoins. En se démarquant de la 
concurrence, Alltech est en forte expansion, elle affiche en 2018 un chiffre d’affaires 6M€.

\subsection {Historique de Alltech}

Aujourd’hui, Alltech compte plus de 100 collaborateurs (Mars 2019) à travers la France pour une équipe d’une dizaine de 
commerciaux. Alltech Consulting s’est implanté rapidement dans les grandes villes françaises. La première agence a été 
créée à Bordeaux, puis sont venues celles de Niort (début 2017) et Toulouse (fin 2017) suivies de près par Nantes, 
Ile-de-France et la région PACA dans l’année 2018. Alltech Consulting dispose désormais de 6 agences sur la France.

\subsection {Services}

En tant qu’Entreprise du Service Numérique, Alltech travaille avec différents clients dans des environnements très 
variés (grande distribution, e-commerce, administrations publiques, startups, etc.). Alltech possède un Centre de 
Services offshore. Ce Centre de Service permet de développer les outils internes mais aussi certains projets de nos 
clients. Alltech développe en parallèle un Centre de Services à Bordeaux afin d’avoir la capacité de répondre à des 
projets internes ou client. Guillaume Cochard s’occupe du pilotage de ces deux centres de services.
Alltech dispose aussi d’un pôle incubateur (INCUBE). Celui-ci donne la possibilité aux consultants de la société de 
mener un projet dans le but de créer une startup. Cela permet de libérer la créativité et le potentiel d’innovation des 
consultants et d’offrir de nouvelles opportunités de croissance pour Alltech.

\subsection{Structure Organisationnelle}

La structure d’Alltech a évolué au fur et à mesure des années. À sa création, le nom était Alltech Consulting. À la 
suite de l’acquisition d’agences supplémentaires, c’est devenu Alltech Group Holding. Enfin, après la création de 
l’agence Ile-de-France, Alltech s’est divisée en deux groupe : Alltech Nouvelle Aquitaine et Alltech Ile-de-France.